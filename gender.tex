\documentclass[20pt, a4]{article}
\usepackage{extsizes}
\usepackage{amsmath}
\begin{document}
What is gender? \\
Firstly, I will define a few terms:
\begin{enumerate}
	\item Base gender: a gender which cannot be decomposed i.e. woman cannot be split into any baser genders, but bigender is composed of two other genders
	\item Association value: a value between -1 and 1, which corresponds to how strongly someone identified with a base gender i.e. 1 is complete association and -1 is complete dissociation, and 0 is apathy
\end{enumerate}
Using these terms, I will define gender to be a function which takes a base gender and a time value as a parameter, and returns a pdf which describes the behaviour of the association value. \\
The reason for the pdf is to account for chaotic genders, and the time value allows for genders to change with time. \\
I will now outline some genders which are easy to define with this methodology:
\begin{enumerate}
	\item Pangender: This simply returns a translation of the dirac delta function such that the peak is stationated at +1 for every for every input
	\item Agender: Much like pangender, except it returns the dirac delta function without any translation applied
\end{enumerate}
The purpose of this methodology is not to prescribe someone a gender which is defined mathematically, but rather to provide a means of empowerment whereby people can represent their own gender identity through mathematics rather than through natural language
\end{document}
